%Caso de uso 1.3.4 
\begin{UseCase} {CU1.3.4}{Modificar venta del historial}{
	Permite que el empleado pueda modificar una venta para devoluciones de articulos a inventario.
}

\UCitem{Versión}{1.0}
\UCccsection{Administración}
\UCccitem{Autor}{EduardoAlfonso Rivera}
\UCccitem{Evaluador}{}
\UCccitem{Operación}{}
\UCccitem{Prioridad}{Alta}
\UCccitem{Complejidad}{Alta}
\UCccitem{Volatilidad}{Baja}
\UCccitem{Madurez}{Alta}
\UCccitem{Estatus}{Edición}
\UCccitem{Fecha del último estatus}{17 de diciembre del 2021}

% Copie y pegue este bloque tantas veces como revisiones tenga el caso de uso.
% Esta sección la debe llenar solo el Revisor
% --------------------------------------------------------

% Revisión Versión 
% Anote la versión que se revisó
\UCccsection{Revisión Versión 0.1 }

% Fecha
% Anote la fecha en que se terminó la revisión
\UCccitem{Fecha}{} 

% Evaluador
% Coloque el nombre completo de quien realizó la revisión
\UCccitem{Evaluador}{}

% Resultado
% Coloque la palabra que mas se apegue al tipo de acción que el analista debe realizar
\UCccitem{Resultado}{}

% Observaciones
% Liste los cambios que debe realizar el Analista.
\UCccitem{Observaciones}{}

% --------------------------------------------------------
	
\UCsection{Atributos}
	\UCitem{Actor(es)}
	{
		\cdtRef{Actor:RT}{Empleado}
	}
	\UCitem{Propósito}
	{
		Realizar la devolución de articulos, agregar los articulos devueltos al inventario y descontar el valor de los articulos a la venta.
	}
	\UCitem{Entradas}
	{
		\UCli Estatus de la venta.
		\UCli Nombre y número de articulo a devolver.
	}
	\UCitem{Salidas}
	{
		\UCli Estatus de la venta.
		\UCli Nombre de articulos vendidos.
		\UCli Cantidad de articulos vendidos.
		\UCli Precio por articulos.
		\UCli Número de ticket o código de venta.
		\UCli Fecha y hora de venta.
		\UCli Nombre de empleado que atendió la venta.
		\UCli Total de venta modificada.
	}

	\UCitem{Precondiciones}
	{
		\UCli Empleado debió de haber realizado la busqueda de la venta a modificar del historial.
		\UCli Se tomará en cuenta la RN018 para determinar si el articulo puede ser devuelto.
		\UCli El estatus de la venta a modificar debe ser: realizada o modificada.
	}
	\UCitem{Postcondiciones}
	{
		Ninguna
	}
	\UCitem{Reglas de negocio}
	{
		\UCli RN021: No se podrán realizar devoluciones de ventas con estatus de eliminadas.
		\UCli RN022: La modificación de una venta dentro del historial de ventas solo se podrá reducir articulos tomandolos como una devolución, si se deseea aumentar articulos a la venta no se podrá, se debe realizar una venta nueva.
		\UCli RN018: Solo se podrán realizar devoluciones si el medicamento o articulo se encuentra en perfectas condiciones, en el caso de medicamentos de uso exclusivo con receta deberán estar sellados.
		\UCli RN027: Solo se podrán realizar devoluciones en el mismo mes de la venta.
		\UCli RN025: Toda venta realizada en la farmacia debera tener asociado el nombre del empleado que realiza la venta, así mismo, si un empleado realiza una modificación en una venta, el nombre del empleado se actualiza por el empleado que realizo la modificación.
		
	}
	\UCitem{Errores}
	{
		Ninguna
	}
	\UCitem{Tipo}{}
\end{UseCase}

%Trayectoria principal

\begin{UCtrayectoria}

	\UCpaso [\UCactor]  Realiza busqueda de una venta en historial. <<include>> CU1.3.1 Busqueda de una venta en historial.
	\UCpaso [\UCactor]  Da click en el icono de lapiz en la pantalla Gestionar venta del historial. 
	\UCpaso [\UCsist] Obtiene los detalles de la venta en la BD. [Trayectoria A]
	\UCpaso [\UCsist] Muestra la pantalla2 Modificar venta del historial, con la información obtenida de los detalles de la venta.
	\UCpaso [\UCactor] Disminuye la cantidad de articulos de la venta.
	\UCpaso [\UCactor] Da click en el botón terminar. \cdtButton{Terminar}
	\UCpaso [\UCsist] Valida que los articulos se pueden devolver en base a la RN027. [Trayectoria B]
	\UCpaso [\UCsist] Muestra una alerta con el mensaje1 "¿Estás seguro de modificar esta venta? Esta venta será modificada del historial y aparecerá en estatus modificada."
	\UCpaso [\UCactor] Selecciona la opción aceptar. \cdtButton{Aceptar} [Trayectoria C]
	\UCpaso [\UCactor] Quita los articulos de la venta.
	\UCpaso [\UCsist] Notifica a inventario el ingreso de los productos devueltos.
	\UCpaso [\UCsist] Devuelve el número de articulos modificados de la venta al inventario.
	\UCpaso [\UCsist] Actualiza el total de la venta.
	\UCpaso[\UCsist] Descuenta la diferencia del monto de la venta actualizada al calculo de venta mensual.
	\UCpaso [\UCsist] Actualiza el nombre del empleado asociado a la venta en base a la RN025.
	\UCpaso [\UCsist] Modifica el estatus de la venta a Modificada con base en el DE1.
	\UCpaso [\UCsist] Regresa a la pantalla Gestionar venta del historial.
	\UCpaso [\UCsist] Muestra notificación con mensaje2: "Se modificó el estatus de la venta con el código, ha sido modificado el historial."
\end{UCtrayectoria}


% Trayectorias alternativas
% Trayectoria alternativa A
\begin{UCtrayectoriaA}{A}{ No puede realizar la consulta de la venta en la base de datos.}
	\UCpaso [\UCsist] Notifica con mensaje4: "No se realizo la consulta en la base de datos".
	\UCpaso [\UCsist] Se regresará a el paso 1 de la TP.
\end{UCtrayectoriaA}
%Trayectoria alternativa B
\begin{UCtrayectoriaA}{B}{ No cumple la RN027: La devolución no se hace en el mismo mes de la venta.}
	\UCpaso [\UCsist] Notifica con mensaje4: "No se acepta la devolución porque no cumple la RN027".
	\UCpaso [\UCsist] Se regresará a el paso 1 de la TP.
\end{UCtrayectoriaA}
%Trayectoria alternativa C
\begin{UCtrayectoriaA}{C}{Empleado selecciona la opcion cancelar.}
	\UCpaso [\UCactor] Selecciona la opción cancelar.\cdtButton{Cancelar}
	\UCpaso [\UCsist] Notifica con mensaje3: "No se modifica el estatus de la venta con el código, No ha sido modificada el historial."
	\UCpaso [\UCsist] Regresa a el paso 1 de la TP.

\end{UCtrayectoriaA}