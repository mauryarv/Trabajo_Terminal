\begin{UseCase}{CUTI5.1-8}{Agregar proveedor} {%
	
El empleado agrega en el sistema un nuevo proveedor a través de un formulario que el sistema proporciona.
	
}
	
\UCitem{Versión}{1.0}
\UCccsection{Administración}
\UCccitem{Autor}{Reyes Vaca Mauricio Alberto}
\UCccitem{Evaluador}{}
\UCccitem{Operación}{}
\UCccitem{Prioridad}{Alta}
\UCccitem{Complejidad}{Media}
\UCccitem{Volatilidad}{Baja}
\UCccitem{Madurez}{Alta}
\UCccitem{Estatus}{Edición}
\UCccitem{Fecha del último estatus}{20 de noviembre del 2021}

%% Copie y pegue este bloque tantas veces como revisiones tenga el caso de uso.
%% Esta sección la debe llenar solo el Revisor
% %--------------------------------------------------------
	%   % Revisión Versión (Anote la versión que se revisó.)
\UCccsection{Revisión Versión 0.1 }
% 	% FECHA: Anote la fecha en que se terminó la revisión.
\UCccitem{Fecha}{} 
% 	% EVALUADOR: Coloque el nombre completo de quien realizó la revisión.
\UCccitem{Evaluador}{}
% 	% RESULTADO: Coloque la palabra que mas se apegue al tipo de acción que el analista debe realizar.
\UCccitem{Resultado}{}
% 	% OBSERVACIONES: Liste los cambios que debe realizar el Analista.
\UCccitem{Observaciones}{
	\begin{UClist}
		%\RCitem{PC1}{\DONE{En la pantalla falta el botón del ojito}}{18 de Abril del 2018}
	\end{UClist}		
}

% %--------------------------------------------------------
	
\UCsection{Atributos}
	\UCitem{Actor(es)}{
		\cdtRef{Actor:AT}{Administrador}
	}
	\UCitem{Propósito}{
		Agregar al sistema un nuevo proveedor.
	}
	\UCitem{Entradas}{
		\begin{UClist}
			Información básica del proveedor.			
			\UCli Nombre: (\cdtRef{Alumno:primerApellido}{Primer apellido}, \cdtRef{Alumno:segundoApellido}{Segundo apellido} y \cdtRef{Alumno:nombre}{Nombre}).
			
			\UCli \cdtRef{tesis:Título de la empresa}{Empresa}.
			
			\UCli \cdtRef{tesis:Telefono de contacto}{Telefono}.
			
			\UCli \cdtRef{tesis:Correo electronico}{Correo}.
		\end{UClist}
	}
	\UCitem{Salidas}{

	}

	\UCitem{Precondiciones}{
		Ninguna
	}
	\UCitem{Postcondiciones}{
		Ninguna
	}
	\UCitem{Reglas de negocio}{
		Ninguna
	}
	\UCitem{Errores}{
		Ninguna
	}
	\UCitem{Tipo}{Primario}
\end{UseCase}

\begin{UCtrayectoria}
	
	\UCpaso[\UCactor] Selecciona \cdtButton{Proveedores} en la interfaz Menu principal.
	\UCpaso[\UCsist] Muestra la interfaz Menu proveedores.
	\UCpaso[\UCactor] Selecciona \cdtButton{Gestionar proveedor}.
	\UCpaso[\UCsist] Obtiene todos los datos de los proveedores de la base de datos. \refTray{E}
	\UCpaso[\UCsist] Muestra la interfaz con los datos anteriores.
	\UCpaso[\UCactor] Selecciona el icono + para poder agregar un proveedor.
	\UCpaso[\UCsist] Muestra la intrefaz de Agregar proveedores.\refTray{F}
          \UCpaso[\UCactor] Ingresa el nombre de la empresa a donde pertenece el proveedor.
	\UCpaso[\UCactor] Ingresa el nombre del proveedor.
	\UCpaso[\UCactor] Ingresa el telefono del proveedor.
	\UCpaso[\UCactor] Ingresa el correo del proveedor.
	\UCpaso[\UCactor] Da clic en el botòn  \cdtButton{Guardar}. \refTray{A}
	\UCpaso[\UCsist] Verifica que el telefono ingresado sea válido de acuerdo con la regla de negocio RN1.\refTray{B} 
	\UCpaso[\UCsist] Verifica que el correo ingresado sea válido de acuerdo con la regla de negocio RN2.\refTray{C}
	\UCpaso[\UCsist] Verifica que el proveedor no este repetido. \refTray{D}
	\UCpaso[\UCsist] Asigna id al proveedor.
	\UCpaso[\UCsist] Guarda los cambios en la base de datos.
	\UCpaso[\UCsist] Envía mensaje 6: "El proveedor fue agregado con éxito".
	\UCpaso[\UCsist] Vacía el formulario.
	\UCpaso[\UCsist] Regresa a la pantalla de gestiòn de proveedores con la información registrada.
	
\end{UCtrayectoria}


%...........::::::::::::TRAYECTORIAS ALTERNATIVAS::::::::::::::........
%-------------------------Trayectoria A------------------------------
\begin{UCtrayectoriaA}{A}{Campos incompletos}
	\UCpaso[\UCsist] Envía el mensaje 1: "Debes ingresar la información requerida en todos los campos.			
	\UCpaso[] Regresa al paso 8 de la trayectoria principal.
\end{UCtrayectoriaA}

%-------------------------Trayectoria B------------------------------
\begin{UCtrayectoriaA}{B}{El número telefonico ingresado no es válido}
	\UCpaso[\UCsist] Envía el mensaje 2: "Debes ingresar un número telefonico valido."			
	\UCpaso[] Regresa al paso 8 de la trayectoria principal.
\end{UCtrayectoriaA}

%-------------------------Trayectoria C------------------------------
\begin{UCtrayectoriaA}{C}{El correo ingresado no es valido}
	\UCpaso[\UCsist] Envía el mensaje 3: "Debes ingresar un correo valido".			
	\UCpaso[] Regresa al paso 8 de la trayectoria principal.
\end{UCtrayectoriaA}

%-------------------------Trayectoria D------------------------------
\begin{UCtrayectoriaA}{D}{Proveedor previamente registrado}
	\UCpaso[\UCsist] iclude: CU 1.4. Consultar proveedor. 
	\UCpaso[\UCsist] Envía el mensaje 4: "El proveedor que ingresó ya había sido registrado con anterioridad, por favor ingrese uno nuevo."			
	\UCpaso[] Regresa al paso 8 de la trayectoria principal.
\end{UCtrayectoriaA}

%-------------------------Trayectoria E------------------------------
\begin{UCtrayectoriaA}{E}{Base de Datos caída}
	\UCpaso[\UCsist] Envía el mensaje 5: "Base de Datos no disponible, por favor intente más tarde".			
	\UCpaso[] Regresa al paso 2 de la trayectoria principal.
\end{UCtrayectoriaA}
%-------------------------Trayectoria F------------------------------
\begin{UCtrayectoriaA}{F}{Click en regresar}
	\UCpaso[] Regresa al paso 2 de la trayectoria principal.
\end{UCtrayectoriaA}