% Caso de uso 10 -------------------------------------------------------------------------------
\begin{UseCase} {CU 10}{Enviar Mensaje}{
	El alumno ya sea conductor o pasajero, puede comunicarse con otro alumno para así acordar alguna circunstancia.
}

% Información del caso de uso ---------------------------------------------------------------------

\UCitem{Versión}{1.0}
\UCccsection{Administración}
\UCccitem{Autor}{Sevillano Mendoza Victor Manuel}
\UCccitem{Evaluador}{Sevillano Mendoza Victor Manuel}
\UCccitem{Operación}{Mensajeria}
\UCccitem{Prioridad}{Media}
\UCccitem{Complejidad}{Alta}
\UCccitem{Volatilidad}{Baja}
\UCccitem{Madurez}{Baja}
\UCccitem{Estatus}{Edición}
\UCccitem{Fecha del último estatus}{16 de abril}

% Revisión ----------------------------------------------------------------------------------------
% Copie y pegue este bloque tantas veces como revisiones tenga el caso de uso.
% Esta sección la debe llenar solo el Revisor

% Revisión Versión % Anote la versión que se revisó
\UCccsection{Revisión Versión 0.1 }

% Fecha % Anote la fecha en que se terminó la revisión
\UCccitem{Fecha}{16 de abril} 

% Evaluador % Coloque el nombre completo de quien realizó la revisión
\UCccitem{Evaluador}{Sevillano Mendoza Victor Manuel}

% Resultado % Coloque la palabra que mas se apegue al tipo de acción que el analista debe realizar
\UCccitem{Resultado}{Redacción y Ortografía}

% Observaciones % Liste los cambios que debe realizar el Analista.
\UCccitem{Observaciones}{Ninguna}

% Atributos ---------------------------------------------------------------------------------------
	
\UCsection{Atributos}

	\UCitem{Actores}{
		\cdtRef{Actor: Alumno}{Alumno}.
	}

	\UCitem{Propósito}{
		Que los alumnos puedan mantenerse comunicados a través de la misma aplicación.
	}

	\UCitem{Entradas}{
		\begin{UClist}			
			\UCli Mensaje.
		\end{UClist}
	}

	\UCitem{Salidas}{
		\begin{UClist}			
			\UCli Teclado.
			\UCli Conversación.
		\end{UClist}
	}

	\UCitem{Precondiciones}{
		\begin{UClist}
			\UCli Se haya iniciado una conversación con anterioridad con el respectivo alumno.
		\end{UClist}
	}

	\UCitem{Postcondiciones}{		
		Ninguna.
	}

	\UCitem{Reglas de negocio}{
		\begin{UClist}
			\UCli Permitir enviar y recibir mensajes.
		\end{UClist}
	}

	\UCitem{Errores}{
		\begin{UClist}
			\UCli Mensaje no pudo enviarse.
		\end{UClist}
	}

	\UCitem{Tipo}{
		
	}
\end{UseCase}

%Trayectoria principal ----------------------------------------------------------------------------

\begin{UCtrayectoria}
	
	\UCpaso [\UCactor]	Selecciona la caja de texto en la pantalla de conversación. 
	\UCpaso [\UCsist]Muestra el teclado virtual y otras opciones.
	\UCpaso [\UCactor]Escribe el mensaje y envia seleccionando \cdtButton{Enviar}.\refTray{A}
	\UCpaso [\UCsist]Actualiza la conversación con el mensaje enviado.\refTray{B}
	\UCpaso [\UCactor]	Selecciona \cdtButton{Volver} en la conversación. 
	\UCpaso [\UCsist]Muestra la la pantalla de perfil del otro alumno.
\end{UCtrayectoria}


% Trayectorias alternativas -----------------------------------------------------------------------

% Trayectoria A
\begin{UCtrayectoriaA}{A}{Error al enviar el mensaje}
	\UCpaso [\UCsist]Muestra mensaje: \cdtRef{MSG}{No se pudo enviar el mensaje}.
	\UCpaso [\UCactor]	Selecciona \cdtButton{OK}
	\UCpaso Va al paso 3 de la trayectoria principal.
\end{UCtrayectoriaA}

% Trayectoria B
\begin{UCtrayectoriaA}{B}{Seguir la conversación}
	\UCpaso [\UCsist]En la pantalla de conversación se muestra la caja de texto.
	\UCpaso [\UCactor]	Selecciona la caja de texto para poder enviar otro mensaje.
	\UCpaso Regresa al paso 2 de la trayectoria principal.
\end{UCtrayectoriaA}
