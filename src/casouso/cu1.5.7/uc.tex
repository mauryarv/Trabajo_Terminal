\begin{UseCase}{CU 1.5.7}{Seleccionar Proveedor para Producto} {
	Seleccionar proveedor para un producto a la hora de llenar el formulario de Datos del Producto. Abre una ventana que muestra la interfaz "Seleccionar Proveedor" con los datos de los proveedores y un icono al lado de sus datos para seleccionarlo. Es indispensable que se escoja un proveedor para que el formulario pueda enviarse.
}
	
\UCitem{Versión}{1.0}
\UCccsection{Administración}
\UCccitem{Autor}{Diego Armando Hernández Penilla}
\UCccitem{Evaluador}{}
\UCccitem{Operación}{}
\UCccitem{Prioridad}{Media}
\UCccitem{Complejidad}{Baja}
\UCccitem{Volatilidad}{Baja}
\UCccitem{Madurez}{Alta}
\UCccitem{Estatus}{Edición}
\UCccitem{Fecha del último estatus}{20 de diciembre del 2021}

%% Copie y pegue este bloque tantas veces como revisiones tenga el caso de uso.
%% Esta sección la debe llenar solo el Revisor
% %--------------------------------------------------------
	%   % Revisión Versión (Anote la versión que se revisó.)
\UCccsection{Revisión Versión 0.1 }
% 	% FECHA: Anote la fecha en que se terminó la revisión.
\UCccitem{Fecha}{} 
% 	% EVALUADOR: Coloque el nombre completo de quien realizó la revisión.
\UCccitem{Evaluador}{}
% 	% RESULTADO: Coloque la palabra que mas se apegue al tipo de acción que el analista debe realizar.
\UCccitem{Resultado}{}
% 	% OBSERVACIONES: Liste los cambios que debe realizar el Analista.
\UCccitem{Observaciones}{
	\begin{UClist}
		%\RCitem{PC1}{\DONE{En la pantalla falta el botón del ojito}}{18 de Abril del 2018}
	\end{UClist}		
}

% %--------------------------------------------------------
	
\UCsection{Atributos}
	\UCitem{Actor(es)}{
		\cdtRef{Actor: Empleado}{Empleado}
	}
	\UCitem{Propósito}{
		Seleccionar el proveedor del producto para registrarlo.
	}
	\UCitem{Entradas}{
		Ninguna
	}
	\UCitem{Salidas}{
		Mensaje 1: " No se logró conectar a la base de datos".
		Mensaje 2: "No se encuentran proveedores registrados, registrar uno".
	}

	\UCitem{Precondiciones}{
		Haber registrado un proveedor.
	}
	\UCitem{Postcondiciones}{
		El Empleado podrá terminar de registrar un producto ya que la selección de su proveedor es un campo obligatorio.
	}
	\UCitem{Reglas de negocio}{
		Ninguna.
	}
	\UCitem{Errores}{
		El Sistema no puede acceder a la Base de Datos.
		No hay proveedores registrados.
	}
	\UCitem{Tipo}{
		Include	
	}

\end{UseCase}

\begin{UCtrayectoria}
	
	\UCpaso[\UCactor] En el formulario Registrar Producto o el de Modificar Producto, hacer clic en la flechita al lado del campo de "Proveedor".
	
	\UCpaso[\UCsist] Se conecta a la base de datos. [Trayectoria A]

	\UCpaso[\UCsist] Muestra en otra ventata la interfaz "Seleccionar Proveedor".

	\UCpaso[\UCsist] Obtiene los proveedores de la base de datos de 30 en 30.  [Trayectoria B].

	\UCpaso[\UCsist] Muestra la lista de proveedores.

	\UCpaso[\UCactor] Hace clic en el símbolo de manita señalando hacia arriba para seleccionar al proveedor que desea.

	\UCpaso[\UCsist] Llena el campo con el nombre del proveedor seleccionado.
	
\end{UCtrayectoria}

%...........::::::::::::TRAYECTORIAS ALTERNATIVAS::::::::::::::........
%-------------------------Trayectoria A------------------------------
\begin{UCtrayectoriaA}{A}{Sistema no puede acceder a la Base de Datos.}
	
	\UCpaso[\UCsist] Envía el mensaje 1: " No se logró conectar a la base de datos".
	
	\UCpaso[\UCactor] Hace clic en "Confirmar".

	\UCpaso[\UCsist] Deja de mostrar el mensaje.

	\UCpaso[]  Permanece en el formulario Registrar Producto (o Modificar Producto).

\end{UCtrayectoriaA}

%-------------------------Trayectoria B------------------------------
\begin{UCtrayectoriaA}{B}{No hay proveedores registrados.}
	
	\UCpaso[\UCsist] Envía el mensaje 2: "No se encuentran proveedores registrados, registrar uno".

	\UCpaso[\UCactor] Hace clic en "Confirmar".

	\UCpaso[\UCsist] Deja de mostrar el mensaje.

	\UCpaso[\UCsist] Cierra la ventana de la interfaz "Seleccionar Proveedor"

	\UCpaso[]  Permanece en el formulario Registrar Producto (o Modificar Producto).

\end{UCtrayectoriaA}