% Caso de uso 26 -------------------------------------------------------------------------------
\begin{UseCase} {CU 26}{Crear historial de usuario}{
	El administrador crea el historial del usuario a partir de las faltas que este comete.
}

% Información del caso de uso ---------------------------------------------------------------------

\UCitem{Versión}{1.0}
\UCccsection{Administración}
\UCccitem{Autor}{Sevillano Mendoza Victor Manuel}
\UCccitem{Evaluador}{Sevillano Mendoza Victor Manuel}
\UCccitem{Operación}{Editar}
\UCccitem{Prioridad}{Media}
\UCccitem{Complejidad}{Alta}
\UCccitem{Volatilidad}{Baja}
\UCccitem{Madurez}{Baja}
\UCccitem{Estatus}{Edición}
\UCccitem{Fecha del último estatus}{16 de abril}

% Revisión ----------------------------------------------------------------------------------------
% Copie y pegue este bloque tantas veces como revisiones tenga el caso de uso.
% Esta sección la debe llenar solo el Revisor

% Revisión Versión % Anote la versión que se revisó
\UCccsection{Revisión Versión 0.1 }

% Fecha % Anote la fecha en que se terminó la revisión
\UCccitem{Fecha}{16 de abril} 

% Evaluador % Coloque el nombre completo de quien realizó la revisión
\UCccitem{Evaluador}{Sevillano Mendoza Victor Manuel}

% Resultado % Coloque la palabra que mas se apegue al tipo de acción que el analista debe realizar
\UCccitem{Resultado}{Redacción y Ortografía}

% Observaciones % Liste los cambios que debe realizar el Analista.
\UCccitem{Observaciones}{Ninguna}

% Atributos ---------------------------------------------------------------------------------------
	
\UCsection{Atributos}

	\UCitem{Actores}{
		\cdtRef{Actor: Administrador}{Administrador}.
	}

	\UCitem{Propósito}{
		Tener registro de las faltas cometidos por los alumno en la aplicación y poder dar sansiones justas.
	}

	\UCitem{Entradas}{
		\begin{UClist}			
			\UCli Nombre de usuario.
			\UCli Registro.
		\end{UClist}
	}

	\UCitem{Salidas}{
		\begin{UClist}			
			\UCli Historial del alumno.
		\end{UClist}
	}

	\UCitem{Precondiciones}{
		\begin{UClist}			
			\UCli El alumno comete una falta.
		\end{UClist}
	}

	\UCitem{Postcondiciones}{		
		Ninguna
	}

	\UCitem{Reglas de negocio}{
		\begin{UClist}
			\UCli El alumno debe estar inscrito en ESCOM.
			\UCli Si el alumno hace una falta obtendra una sanción.
		\end{UClist}
	}

	\UCitem{Errores}{
		Ninguno.
	}

	\UCitem{Tipo}{
		
	}
\end{UseCase}

%Trayectoria principal ----------------------------------------------------------------------------

\begin{UCtrayectoria}	
	\UCpaso [\UCactor]	Selecciona \cdtButton{Suspender} en la pantalla de Suspender Cuenta. 
	\UCpaso [\UCsist]Crea el historial del alumno colocando el registro de la suspensión.\refTray{A}
	\UCpaso Va al paso 6 del CU 25.
\end{UCtrayectoria}


% Trayectorias alternativas -----------------------------------------------------------------------

% Trayectoria A
\begin{UCtrayectoriaA}{A}{El historial ya existe}
	\UCpaso [\UCsist]Actualiza el historial con el nuevo registro hecho.
	\UCpaso Va al paso 3 de la trayectoria principal.
\end{UCtrayectoriaA}

