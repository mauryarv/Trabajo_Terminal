\begin{UseCase}{CU14}{Cancelar el registro de un aspirante} {  %Eduardo Riv
	
	Permite al administrador Cancelar el registro del aspirante de acuerdo a las RN para la admisión.


}
	
\UCitem{Versión}{1.0}
\UCccsection{Admisiones}
\UCccitem{Autor}{Eduardo Alfonso Rivera Morelos}
\UCccitem{Evaluador}{}
\UCccitem{Operación}{}
\UCccitem{Prioridad}{Alta}
\UCccitem{Complejidad}{Media}
\UCccitem{Volatilidad}{Baja}
\UCccitem{Madurez}{Alta}
\UCccitem{Estatus}{Edición}
\UCccitem{Fecha del último estatus}{28 de Octubre de 2021}

%% Copie y pegue este bloque tantas veces como revisiones tenga el caso de uso.
%% Esta sección la debe llenar solo el Revisor
% %--------------------------------------------------------
	%   % Revisión Versión (Anote la versión que se revisó.)
\UCccsection{Revisión Versión 0.1 }
% 	% FECHA: Anote la fecha en que se terminó la revisión.
\UCccitem{Fecha}{} 
% 	% EVALUADOR: Coloque el nombre completo de quien realizó la revisión.
\UCccitem{Evaluador}{}
% 	% RESULTADO: Coloque la palabra que mas se apegue al tipo de acción que el analista debe realizar.
\UCccitem{Resultado}{}
% 	% OBSERVACIONES: Liste los cambios que debe realizar el Analista.
\UCccitem{Observaciones}{
	\begin{UClist}
		%\RCitem{PC1}{\DONE{En la pantalla falta el botón del ojito}}{18 de Abril del 2018}
	\end{UClist}		
}

% %--------------------------------------------------------
	
\UCsection{Atributos}
	\UCitem{Actor(es)}
	{
		\cdtRef{Actor:RT}{Administrador}
	}
	\UCitem{Propósito}
	{
		Cancelar el registro de un aspirante por infligir las reglas de negocio
	}
	\UCitem{Entradas}
	{
		\UCli Folio de aspirante
	}
	\UCitem{Salidas}
	{
		\UCli Nombre
		\UCli Fecha de nacimiento
		\UCli Telefono
		\UCli Estatus: Cancelado.
	}

	\UCitem{Precondiciones}
	{
		\UCli Que exista un registo previo
	}
	\UCitem{Postcondiciones}
	{
		Ninguna
	}
	\UCitem{Reglas de negocio}
	{
		\UCli El aspirante no debe hacer trampa. No pueden existir más de un registro por aspirante. Se cancela el registro por petición del alumno o infringe una regla de negocio. 
	}
	\UCitem{Errores}
	{
		Ninguna
	}
	\UCitem{Tipo}{}
\end{UseCase}

%Trayectoria principal

\begin{UCtrayectoria}
		
	

	
	
	\UCpaso [\UCactor]Administrador: Selecciona la opción “Cancelar el registro del aspirante”.
	\UCpaso [\UCsist]Incluye: CU “Consultar los datos de un aspirante”.
	\UCpaso [\UCactor]Selecciona opción consultar registro.
	\UCpaso [\UCsist]Muestra un recuadro donde se le pide al aspirante su folio.
	\UCpaso [\UCactor]Ingresa su folio dentro del recuadro correspondiente
	\UCpaso [\UCactor]Selecciona el botón \cdtButton{Buscar}
	\UCpaso [\UCsist]Busca al usuario.
	\UCpaso [\UCsist]Muestra los datos del aspirante registrado.
	\UCpaso [\UCsist]Notifica mensaje 1: Consulta realizada.
	\UCpaso [\UCsist]Sistema: Muestra la interfaz  i.3.a “Cancelar registro de sustentante”
	\UCpaso [\UCactor]Administrador: Pulsa el botón “Cancelar registro” \cdButton{Cancelar}
	\UCpaso [\UCsist]Sistema: Muestra mensaje m.5.a “Seguro que desea cancelar los datos”
	\UCpaso [\UCsist]Sistema: Muestra botones “Aceptar” y “Cancelar” .
	\UCpaso [\UCactor]Selecciona el botón “Aceptar” [Trayectoria Alternativa A] \cdButton{Aceptar}
	\UCpaso [\UCsist]Sistema: Actualiza el estado del registro.
	\UCpaso [\UCsist]Sistema: Muestra el mensaje m.9.a  “Registro cancelado”.
\end{UCtrayectoria}


%...........::::::::::::TRAYECTORIAS ALTERNATIVAS::::::::::::::........
% Trayectorias alternativas
% Trayectoria alternativa A

\begin{UCtrayectoriaA}{A}{El administrador selecciona el botón “Cancelar” }
	\UCpaso [\UCactor]Selecciona el botón “Cancelar”  \cdButton{Cancelar}
	\UCpaso [\UCsist] Muestra mensaje 2:  “Proceso Cancelado”.
	\UCpaso [\UCsist] Regresamos al paso 2 de la TP.		

\end{UCtrayectoriaA}

% Trayectoria alternativa B


