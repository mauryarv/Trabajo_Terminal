% Caso de uso 23 -------------------------------------------------------------------------------
\begin{UseCase} {CU 23}{Editar Perfil}{
	El alumno puede modificar su información dada a la aplicación.
}

% Información del caso de uso ---------------------------------------------------------------------

\UCitem{Versión}{1.0}
\UCccsection{Administración}
\UCccitem{Autor}{Sevillano Mendoza Victor Manuel}
\UCccitem{Evaluador}{Sevillano Mendoza Victor Manuel}
\UCccitem{Operación}{Editar}
\UCccitem{Prioridad}{Alta}
\UCccitem{Complejidad}{Media}
\UCccitem{Volatilidad}{Alta}
\UCccitem{Madurez}{Baja}
\UCccitem{Estatus}{Edición}
\UCccitem{Fecha del último estatus}{16 de abril}

% Revisión ----------------------------------------------------------------------------------------
% Copie y pegue este bloque tantas veces como revisiones tenga el caso de uso.
% Esta sección la debe llenar solo el Revisor

% Revisión Versión % Anote la versión que se revisó
\UCccsection{Revisión Versión 0.1 }

% Fecha % Anote la fecha en que se terminó la revisión
\UCccitem{Fecha}{16 de abril} 

% Evaluador % Coloque el nombre completo de quien realizó la revisión
\UCccitem{Evaluador}{Sevillano Mendoza Victor Manuel}

% Resultado % Coloque la palabra que mas se apegue al tipo de acción que el analista debe realizar
\UCccitem{Resultado}{Redacción y Ortografía}

% Observaciones % Liste los cambios que debe realizar el Analista.
\UCccitem{Observaciones}{Ninguna}

% Atributos ---------------------------------------------------------------------------------------
	
\UCsection{Atributos}

	\UCitem{Actores}{
		\cdtRef{Actor: Alumno}{Alumno}.
	}

	\UCitem{Propósito}{
		Que el alumno pueda modificar la información de su perfil por si hay cambios que desee hacer.
	}

	\UCitem{Entradas}{
		
		\begin{UClist}			
			\UCli Correo electronico.
			\UCli Nombre de usuario.
			\UCli Contraseña.
			\UCli Nombre completo.
			\UCli Numero de boleta.
		\end{UClist}
	}

	\UCitem{Salidas}{
		\begin{UClist}			
			\UCli Perfil actualizado.
		\end{UClist}
	}

	\UCitem{Precondiciones}{
		\begin{UClist}
			\UCli Iniciar sesión.
		\end{UClist}
	}

	\UCitem{Postcondiciones}{		
		Ninguna
	}

	\UCitem{Reglas de negocio}{
		\begin{UClist}
			\UCli El alumno debe estar inscrito en ESCOM.
		\end{UClist}
	}

	\UCitem{Errores}{
		\begin{UClist}
			\UCli Un campo no llenado.
		\end{UClist}
	}

	\UCitem{Tipo}{
		
	}
\end{UseCase}

%Trayectoria principal ----------------------------------------------------------------------------

\begin{UCtrayectoria}
	
	\UCpaso [\UCactor]	Selecciona \cdtButton{Editar Perfil} en la pantalla del Perfil del alumno. 
	\UCpaso [\UCsist]Muestra la pantalla de edición de perfil que permite modificar los campos.
	\UCpaso [\UCactor]Modifica los campos que se desee y selecciona \cdtButton{Autentificar inscripción}.\refTray{A}
	\UCpaso Va al paso 1 de CU 8.
	\UCpaso [\UCsist]Muestra mensaje: \cdtRef{MSG}{Autentificación correcta}.
	\UCpaso [\UCactor]Selecciona \cdtButton{Guardar Cambios}.\refTray{B}
	\UCpaso [\UCsist]Muestra la pantalla de Perfil del alumno actualizada.
\end{UCtrayectoria}


% Trayectorias alternativas -----------------------------------------------------------------------

% Trayectoria A
\begin{UCtrayectoriaA}{A}{La autentificación no valida}
	\UCpaso [\UCsist]Muestra mensaje: \cdtRef{MSG}{No se pudo autentificar}.
	\UCpaso [\UCactor]	Selecciona \cdtButton{OK}
	\UCpaso Va al paso 3 de la trayectoria principal.
\end{UCtrayectoriaA}

% Trayectoria B
\begin{UCtrayectoriaA}{B}{Algun campo no fue llenado}
	\UCpaso [\UCsist]Muestra mensaje: \cdtRef{MSG}{Campo faltante}.
	\UCpaso [\UCactor]	Ingresa el dato del campo marcado.
	\UCpaso Regresa al paso 5 de la trayectoria principal.
\end{UCtrayectoriaA}
