\begin{UseCase}{CU_RDoc}{Registrar los documentos de un aspirante} { %Villavicencio Kevin
	
	Permite al actor registrar las calificaciones que se obtuvieron en el departamental actual, asignando dicha calificación a los alumnos registrados en la materia.
}
	
\UCitem{Versión}{1.0}
\UCccsection{Administración}
\UCccitem{Autor}{García Hernández Héctor Daniel}
\UCccitem{Evaluador}{}
\UCccitem{Operación}{}
\UCccitem{Prioridad}{Alta}
\UCccitem{Complejidad}{Media}
\UCccitem{Volatilidad}{Baja}
\UCccitem{Madurez}{Alta}
\UCccitem{Estatus}{Edición}
\UCccitem{Fecha del último estatus}{27 de Septiembre del 2018}

%% Copie y pegue este bloque tantas veces como revisiones tenga el caso de uso.
%% Esta sección la debe llenar solo el Revisor
% %--------------------------------------------------------
	%   % Revisión Versión (Anote la versión que se revisó.)
\UCccsection{Revisión Versión 0.1 }
% 	% FECHA: Anote la fecha en que se terminó la revisión.
\UCccitem{Fecha}{} 
% 	% EVALUADOR: Coloque el nombre completo de quien realizó la revisión.
\UCccitem{Evaluador}{}
% 	% RESULTADO: Coloque la palabra que mas se apegue al tipo de acción que el analista debe realizar.
\UCccitem{Resultado}{}
% 	% OBSERVACIONES: Liste los cambios que debe realizar el Analista.
\UCccitem{Observaciones}{
	\begin{UClist}
		%\RCitem{PC1}{\DONE{En la pantalla falta el botón del ojito}}{18 de Abril del 2018}
	\end{UClist}		
}

% %--------------------------------------------------------
	
\UCsection{Atributos}
	\UCitem{Actor(es)}{
		\cdtRef{Actor:RT}{Responsable de Titulación} y \cdtRef{Actor:AT}{Auxiliar de Titulación}
	}
	\UCitem{Propósito}{
		Visualizar la información del examen profesional y generar el acta correspondiente.
	}
	\UCitem{Entradas}{
		Ninguna
	}
	\UCitem{Salidas}{
		
		\begin{UClist}
			Información del examen profesional
			\UCli \cdtRef{Alumno:matricula}{Matrícula}. \ioObtener.
			
			\UCli Nombre: (\cdtRef{Alumno:primerApellido}{Primer apellido}, \cdtRef{Alumno:segundoApellido}{Segundo apellido} y \cdtRef{Alumno:nombre}{Nombre}). \ioObtener.
			
			\UCli \cdtRef{tesis:temaTesis}{Tema} de la tesis: \ioObtener.
			
			\UCli \cdtRef{horarioExamenProfesional:fecha}{Fecha} del examen: \ioObtener.
			
			\UCli \cdtRef{horarioExamenProfesional:hora}{Hora} del examen: \ioObtener.
			
			\UCli \cdtRef{Salones:nombre}{Salón} del examen: \ioObtener. 
			
			
			Información del resultado:
			\UCli \cdtRef{informacionExamenProfesional:actaEgresado}{Acta y/o Egresado}: \ioObtener. 
			
			\UCli \cdtRef{resultadoExamenProfesional:resultado}{Resultado}: \ioObtener.
			
			\UCli \cdtRef{resultadoExamenProfesional:mencionEspecial}{Mención especial}: \ioObtener.
						
			
			
			
			
			Información para el acta de examen:
			\UCli \cdtRef{horarioExamenProfesional:fecha}{Fecha} del examen profesional. \ioObtener.
			
			\UCli \cdtRef{horarioExamenProfesional:salon}{Salón} del examen profesional. \ioObtener.
			
			\UCli Nombre: (\cdtRef{Alumno:nombre}{Nombre}, \cdtRef{Alumno:primerApellido}{Primer apellido} y \cdtRef{Alumno:segundoApellido}{Segundo apellido}). \ioObtener.
			
			\UCli Sínodo examinador:  (\cdtRef{profesor:tituloTratamiento}{Título de tratamiento}, \cdtRef{profesor:nombre}{Nombre}, \cdtRef{profesor:primerApellido}{Primer apellido} y \cdtRef{profesor:segundoApellido}{Segundo apellido}) del personal interno y (\cdtRef{invitado:nombre}{Nombre}, \cdtRef{invitado:primerApellido}{Primer apellido} y \cdtRef{invitado:segundoApellido}{Segundo apellido}) del personal externo. \ioObtener.						
		\end{UClist}
	}

	\UCitem{Precondiciones}{
		Ninguna
	}
	\UCitem{Postcondiciones}{
		Ninguna
	}
	\UCitem{Reglas de negocio}{
		Ninguna
	}
	\UCitem{Errores}{
		Ninguna
	}
	\UCitem{Tipo}{Secundario, extiende del caso de uso \cdtIdRef{CUTI5.1-2}{Gestionar exámenes profesionales}.}
\end{UseCase}

\begin{UCtrayectoria}
	
	\UCpaso[\UCactor] Presiona el botón \cdtButton {Registrar Calificación} en la interfaz, Gestionar Materia.
	
	\UCpaso[\UCsist] Obtiene el nombre, matricula y calificaciones anteriores al departamental actual.
	
	\UCpaso[\UCactor] \label{RA:Seleccion} Selecciona al alumno que se le va a registrar su calificación.

	\UCpaso[\UCsist] Valida que el alumno se encuentre en estado Activo en la materia. \refTray{A}
	
	\UCpaso[\UCactor] \label{RA:Calificación} Introduce la calificación en el alumno seleccionado. \refTray{B}
	
	\UCpaso[\UCactor] Presiona el botón \cdtButton{Aceptar}. \refTray{C}
	
	\UCpaso[\UCsist] Persiste la calificación asignada al alumno.
	
	\UCpaso[\UCsist] Muestra el mensaje Operación Exitosa, en la interfaz Registrar Calificación.
	
	\UCpaso[\UCactor] Presiona el botón \cdtButton{Regresar}. \refTray{D}
	
	\UCpaso[\UCsist] Muestra la interfaz Gestionar Materia
	
\end{UCtrayectoria}


%...........::::::::::::TRAYECTORIAS ALTERNATIVAS::::::::::::::........
%-------------------------Trayectoria A------------------------------
\begin{UCtrayectoriaA}{A}{El alumno seleccionado se encuentra en estado inactivo}
	\UCpaso[\UCsist] Muestra el mensaje Alumno en estado inactivo.
	
	\UCpaso[\UCactor] Lee el mensaje.
	
	\UCpaso[\UCactor] Presiona el botón \cdtButton{Aceptar}.

	\UCpaso[\UCsist] Regresa al paso \ref{RA:Seleccion} de la trayectoria principal del caso de uso.
	
\end{UCtrayectoriaA}

%...........::::::::::::TRAYECTORIAS ALTERNATIVAS::::::::::::::........
%-------------------------Trayectoria A------------------------------
\begin{UCtrayectoriaA}{B}{El actor introdujo una calificación no válida}
	\UCpaso[\UCsist] Muestra el mensaje Formato de Calificación erróneo.
	
	\UCpaso[\UCactor] Lee el mensaje.
	
	\UCpaso[\UCactor] Presiona el botón \cdtButton{Aceptar}.
	
	\UCpaso[\UCsist] Regresa al paso \ref{RA:Calificación} de la trayectoria principal del caso de uso.
	
\end{UCtrayectoriaA}