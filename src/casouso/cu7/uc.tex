\begin{UseCase}{CU7}{Generar ticket.} { 
	
	 Permite cumplir con las expectativas y mejorar la experiencia de tus clientes. Además, puedes aumentar la productividad, disminuir los tiempos de respuesta y obtener muchos beneficios tangibles para tus clientes.
}
	
\UCitem{Versión}{1.0}
\UCccsection{Administración}
\UCccitem{Autor}{Martin Cipriano Ortiz Chavez}
\UCccitem{Evaluador}{}
\UCccitem{Operación}{}
\UCccitem{Prioridad}{Alta}
\UCccitem{Complejidad}{Media}
\UCccitem{Volatilidad}{Baja}
\UCccitem{Madurez}{Alta}
\UCccitem{Estatus}{Edición}
\UCccitem{Fecha del último estatus}{21 de noviembre del 2021}

%% Copie y pegue este bloque tantas veces como revisiones tenga el caso de uso.
%% Esta sección la debe llenar solo el Revisor
% %--------------------------------------------------------
	%   % Revisión Versión (Anote la versión que se revisó.)
\UCccsection{Revisión Versión 0.1 }
% 	% FECHA: Anote la fecha en que se terminó la revisión.
\UCccitem{Fecha}{} 
% 	% EVALUADOR: Coloque el nombre completo de quien realizó la revisión.
\UCccitem{Evaluador}{}
% 	% RESULTADO: Coloque la palabra que mas se apegue al tipo de acción que el analista debe realizar.
\UCccitem{Resultado}{}
% 	% OBSERVACIONES: Liste los cambios que debe realizar el Analista.
\UCccitem{Observaciones}{
	\begin{UClist}
		%\RCitem{PC1}{\DONE{En la pantalla falta el botón del ojito}}{18 de Abril del 2018}
	\end{UClist}		
}

% %--------------------------------------------------------
	
\UCsection{Atributos}
	\UCitem{Actor(es)}{
		\cdtRef{Actor:RT}{Farmacia}
	}
	\UCitem{Propósito}{
		Visualizar la información del aspirante y gestionar
	}
	\UCitem{Entradas}{
		\begin{UClist}
		Información del Aspirante

			\UCli nombre.
			
			\UCli Compra.
			
			\UCli Caja.
			
									
		\end{UClist}

	}
	\UCitem{Salidas}{
		
		\begin{UClist}
			Cliente
			\UCli Nombre: \ioObtener.
			
			\UCli Dinero. \ioObtener.
			
			\UCli Receta: \ioObtener. 
			
								
		\end{UClist}
	}

	\UCitem{Precondiciones}{
		\UCli Tener una Compra.

		\UCli Caja actualizadas.
	}
	\UCitem{Postcondiciones}{
		Ninguna
	}
	\UCitem{Reglas de negocio}{
		Ninguna
	}
	\UCitem{Errores}{
		Ninguna
	}
	\UCitem{Tipo}{Ninguna}
	\end{UseCase}

\begin{UCtrayectoria}
	
	\UCpaso[\UCactor] Solicitar su pedido.
	
	\UCpaso[\UCsist] Mostrar interfaz Gestión de compras.
	
	\UCpaso[\UCactor] Solicitar datos de compra.

	\UCpaso[\UCsist] Muestra datos de  producto.

	\UCpaso[\UCactor] Seleccionar producto a comprar.
	
	\UCpaso[\UCsist] Valida proveedores que ofrecen dichos productos. \refTray{A}
	
	\UCpaso[\UCsist] Muestra proveedores de producto seleccionado. \refTray{B}

	\UCpaso[\UCactor] Selecciona proveedor.

	\UCpaso[\UCactor] Pulsa tecla de confirmar.
	
	\UCpaso[\UCsist] Validar datos producto proveedor. \refTray{C}
	
	\UCpaso[\UCsist] Mostrar Lista de Productos realizados. \refTray{D}

	\UCpaso[\UCsist] Valida que el sea correcto el producto. \refTray{E}

	\UCpaso[\UCsist] Pedir Cantidad por productos.

	\UCpaso[\UCactor] Ingresar cantidad de productos.

	\UCpaso[\UCsist] Muestra reporte de producto y su respectiva cantidad, Registro exitoso.

	\UCpaso[\UCsist] Genera monto a pagar.
	 
	\UCpaso[\UCactor] Solicita reporte de compra

	\UCpaso[\UCsist] Imprime TICKET.

	\UCpaso[\UCsist] Abandona la interfaz gestionar compras.

	\UCpaso[\UCactor] Salir del sistema.
	
\end{UCtrayectoria}


%...........::::::::::::TRAYECTORIAS ALTERNATIVAS::::::::::::::........
%-------------------------Trayectoria A------------------------------
\begin{UCtrayectoriaA}{A}{El usuario no ingresó los datos obligatorios}
	
	\UCpaso[\UCsist] Envía el mensaje 2: Debes ingresar todos los campos marcados como obligatorios.
	
	\UCpaso[] Regresa al paso 2 de la trayectoria principal.
	
\end{UCtrayectoriaA}

%.......................Trayectoria B..................................
\begin{UCtrayectoriaA}{B}{El CURP no cumple con un formato válido.}
	\UCpaso[\UCsist] Muestra el mensaje 3: El CURP no cumple con un formato válido
			
	\UCpaso[] Regresa al paso 6 de la trayectoria principal.
	
\end{UCtrayectoriaA}

%-------------------------Trayectoria C------------------------------
\begin{UCtrayectoriaA}{C}{Rango de edad del aspirante}
	\UCpaso[\UCsist]Comprueba que tenga una cierta edad para poder ser registrado
			
	\UCpaso[] Regresa al paso 2 de la trayectoria principal.
	
\end{UCtrayectoriaA}
