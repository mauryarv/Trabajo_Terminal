%Caso de uso 1.2.1.3
\begin{UseCase} {CU1.2.1.3}{Devolver Productos seleccionados}{
	El empleado recibe los productos devueltos y quita los productos seleccionados de una venta para: descontar el P. venta del Total de venta y devolver a inventario.
}

\UCitem{Versión}{1.0}
\UCccsection{Administración}
\UCccitem{Autor}{Rivera Morelos Eduardo Alfonso}
\UCccitem{Evaluador}{}
\UCccitem{Operación}{}
\UCccitem{Prioridad}{Alta}
\UCccitem{Complejidad}{Alta}
\UCccitem{Volatilidad}{Baja}
\UCccitem{Madurez}{Alta}
\UCccitem{Estatus}{Edición}
\UCccitem{Fecha del último estatus}{4 Enero del 2022}

% Copie y pegue este bloque tantas veces como revisiones tenga el caso de uso.
% Esta sección la debe llenar solo el Revisor
% --------------------------------------------------------

% Revisión Versión 
% Anote la versión que se revisó
\UCccsection{Revisión Versión 0.1 }

% Fecha
% Anote la fecha en que se terminó la revisión
\UCccitem{Fecha}{} 

% Evaluador
% Coloque el nombre completo de quien realizó la revisión
\UCccitem{Evaluador}{}

% Resultado
% Coloque la palabra que mas se apegue al tipo de acción que el analista debe realizar
\UCccitem{Resultado}{}

% Observaciones
% Liste los cambios que debe realizar el Analista.
\UCccitem{Observaciones}{}

% --------------------------------------------------------
	
\UCsection{Atributos}
	\UCitem{Actor(es)}
	{
		\cdtRef{Actor:RT}{Administrador de la farmacia}
	}
	\UCitem{Propósito}
	{
		Recibir la devolución de articulos, descontar el valor de los articulos a la venta, notificar a inventario de los articulos devueltos y devolver en efectivo el valor de los articulos devueltos de la venta.
	}
	\UCitem{Entradas}
	{
		\UCli Nombre y cantidad de articulo a devolver.
	}
	\UCitem{Salidas}
	{
		\UCli Número de ticket.
		\UCli Fecha y hora de venta.
		\UCli Total de venta.
		\UCli Nombre de productos vendidos.
		\UCli Código de barras de productos vendidos.
		\UCli Descripción de productos vendidos.
		\UCli Cantidad de productos vendidos.
		\UCli P. Venta por producto.
		\UCli Subtotal por producto.
		
	}

	\UCitem{Precondiciones}
	{
		\UCli Cliente debe presentar ticket original de venta.
		\UCli Ciente debe presentar los articulos a devolver fisicamente.
		\UCli Empleado debe tener en efectivo un monto mayor o igual al de la devolución.
		\UCli Empleado Realizo la busqueda de la venta a modificar.
		\UCli Se tomará en cuenta la \cdtRef{RN-018}{Devolución} para determinar si el articulo puede ser devuelto.
	}
	\UCitem{Postcondiciones}
	{
		Ninguna
	}
	\UCitem{Reglas de negocio}
	{
		\UCli \cdtRef{RN-018}{Devolución}
		\UCli \cdtRef{RN-019}{Requisitos para una devolución}
		\UCli \cdtRef{RN-020}{Condición de un producto a devolver}
		\UCli \cdtRef{RN-022}{Fecha limite para devolver un producto}
		\UCli \cdtRef{RN-025}{Actualización de fecha y hora de una venta}
		\UCli \cdtRef{RN-027}{Cliente solicita la venta de un producto que recién devolvio}
		
	}
	\UCitem{Errores}
	{
		Ninguna
	}
	\UCitem{Tipo}{}
\end{UseCase}

%Trayectoria principal

\begin{UCtrayectoria}

	\UCpaso [\UCactor]  Busca la venta que contiene los productos a devolver. Include: \cdtRef{CU1.2.1}{CU1.2.1 Buscar venta por número de ticket}
	\UCpaso [\UCactor] Verifica que el cliente lleve los productos fisicos con base en la regla de negocio  \cdtRef{RN-018}{Devolución}.\refTray{A}
	\UCpaso [\UCactor] Comprueba que el cliente tenga en original y una copia el ticket de la venta con base a la regla de negocio  \cdtRef{RN-019}{Requisitos para una devolución}.\refTray{B} 
\UCpaso [\UCactor] Examina que el producto a devolver se encuentra en optimas condiciones con base en la regla de negocio \cdtRef{RN-020}{Condición de un producto a devolver}.\refTray{C} 


	\UCpaso [\UCactor] Disminuye un producto de la venta oprimiendo el botón \cdtButton{-}.
	\UCpaso [\UCactor] Oprime el botón Confirmar. \cdtButton{Confirmar}
	\UCpaso [\UCsist] Valida que los productos se puedan devolver con base en la regla de negocio \cdtRef{RN-020}{Fecha limite para devolver un producto}.\refTray{D} 
	\UCpaso [\UCsist] Muestra los datos de los productos a devolver de la venta en la pantalla2 \cdtRef{UI}{Devolución|confirmación}
	\UCpaso [\UCsist] Muestra una alerta con el mensaje1: \cdtRef{MSG}{¿Estás seguro de la devolución? Se ingresarán los productos a inventario}.
	\UCpaso [\UCactor] Selecciona la opción aceptar. \cdtButton{Aceptar} \refTray{E}
	\UCpaso [\UCsist] Quita de la venta los productos devueltos.
	\UCpaso [\UCsist] Notifica a inventario el ingreso de los productos devueltos.
	\UCpaso [\UCsist] Ingresan los productos devueltos de la venta al inventario.

	\UCpaso [\UCsist] Calcula la suma de los  P. de venta de cada producto devuelto.
	\UCpaso [\UCsist] Guarda la suma del los P. de venta de los productos devueltos.

	
	\UCpaso[\UCsist] Descuenta la suma del  los P. de venta de los productos devueltos al Total de la venta.
	\UCpaso [\UCsist] Actualiza el total de la venta por el total de la venta con los productos devueltos.
	
	\UCpaso [\UCsist] Actualiza la fecha y hora de venta con base en la regla de negocio \cdtRef{RN-025}{Actualización de fecha y hora de una venta}
	\UCpaso [\UCsist] Actualiza los detalles de la venta sin los articulos devueltos: Nombre de articulos vendidos, Código de barras de articulos vendidos, Descripción de articulos vendidos, Cantidad de articulos vendidos, P. de venta por articulo, Subtotal por articulo.
	
	\UCpaso [\UCsist] Actualiza en la BD el total de la venta, la fecha y hora de venta con base en la reglas de negocio: \cdtRef{RN-025}{Actualización de fecha y hora de una venta} y \cdtRef{RN-014}{Datos esenciales de una venta}.
	\UCpaso [\UCsist] Actualiza en la BD los detalles de la venta: Nombre de articulos vendidos, Código de barras de articulos vendidos, Descripción de articulos vendidos, Cantidad de articulos vendidos, P. de venta por articulo, Subtotal por articulo.

	\UCpaso [\UCsist] Muestra notificación con mensaje2: \cdtRef{MSG}{Devolución realizada exitosamente}.
	\UCpaso [\UCsist] Muestra la suma de los P. venta de los productos devueltos.
	\UCpaso [\UCactor] Guarda la copia del ticket de la venta.
	\UCpaso [\UCactor] Entrega en efectivo al cliente la suma de los P.venta de los productos devueltos.
	\UCpaso [\UCactor] Oprime OK en la alerta \cdtButton{Ok}
	\UCpaso [\UCsist] Se regresará a el paso 2 de la TP del \cdtRef{CU1.2.1}{CU1.2.1 Buscar venta por número de ticket}.

\end{UCtrayectoria}


% Trayectorias alternativas
% Trayectoria alternativa A
\begin{UCtrayectoriaA}{A}{ Cliente no cumple con la regla de negocio  \cdtRef{RN-018}{Devolución} }
	\UCpaso [\UCactor] Da click en el botón Cancelar devolución en la pantalla2 \cdtRef{UI}{Devolución|lista}. \cdtButton{Cancelar devolución}
	\UCpaso [\UCsist] Notifica con mensaje4:\cdtRef{MSG} {Devolución cancelada}.
	\UCpaso [\UCactor] Oprime OK en la alerta \cdtButton{Ok}
	\UCpaso [\UCsist] Se regresará a el paso 2 de la TP del \cdtRef{CU1.2.1}{CU1.2.1 Buscar venta por número de ticket}.
\end{UCtrayectoriaA}

% Trayectoria alternativa B
\begin{UCtrayectoriaA}{B}{ Cliente no cumple con la regla de negocio  \cdtRef{RN-019}{Requisitos para una devolución} }
	\UCpaso [\UCactor] Da click en el botón Cancelar devolución en la pantalla2 \cdtRef{UI}{Devolución|lista}. \cdtButton{Cancelar devolución}
	\UCpaso [\UCsist] Notifica con mensaje4: \cdtRef{MSG}{Devolución cancelada}.
	\UCpaso [\UCactor] Oprime OK en la alerta \cdtButton{Ok}
	\UCpaso [\UCsist] Se regresará a el paso 2 de la TP del \cdtRef{CU1.2.1}{CU1.2.1 Buscar venta por número de ticket}.
\end{UCtrayectoriaA}
% Trayectoria alternativa C
\begin{UCtrayectoriaA}{C}{ El producto a devolver no cumple con la regla de negocio  \cdtRef{RN-020}{Condición de un producto a devolver} }
	\UCpaso [\UCactor] Da click en el botón Cancelar devolución en la pantalla2 \cdtRef{UI}{Devolución|lista}. \cdtButton{Cancelar devolución}
	\UCpaso [\UCsist] Notifica con mensaje4: \cdtRef{MSG}{Devolución cancelada}.
	\UCpaso [\UCactor] Oprime OK en la alerta \cdtButton{Ok}
	\UCpaso [\UCsist] Se regresará a el paso 2 de la TP del \cdtRef{CU1.2.1}{CU1.2.1 Buscar venta por número de ticket}.
\end{UCtrayectoriaA}
% Trayectoria alternativa D
\begin{UCtrayectoriaA}{D}{ El producto a devolver no cumple con la regla de negocio \cdtRef{RN-022}{Fecha limite para devolver un producto} }
	\UCpaso [\UCsist] Notifica con mensaje4:\cdtRef{MSG}{Devolución cancelada}.
	\UCpaso [\UCactor] Oprime Aceptar en la alerta \cdtButton{Aceptar}
	\UCpaso [\UCsist] Se regresará a el paso 2 de la TP del \cdtRef{CU1.2.1}{CU1.2.1 Buscar venta por número de ticket}.
\end{UCtrayectoriaA}
%Trayectoria alternativa E
\begin{UCtrayectoriaA}{E}{Empleado selecciona la opcion cancelar.}
	\UCpaso [\UCactor] Selecciona la opción cancelar.\cdtButton{Cancelar}
	\UCpaso [\UCsist] Notifica con mensaje4:\cdtRef{MSG}{Devolución cancelada}.
	\UCpaso [\UCsist] Regresa a el paso 2 de la TP del \cdtRef{CU1.2.1}{CU1.2.1 Buscar venta por número de ticket}.

\end{UCtrayectoriaA}