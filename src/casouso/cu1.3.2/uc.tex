%Caso de uso 1.3.2
\begin{UseCase} {CU1.3.2}{Consultar los detalles de una venta del historial}{
	Permite visualizar los datos asociados a una venta en la farmacia.
}

\UCitem{Versión}{1.0}
\UCccsection{Administración}
\UCccitem{Autor}{Eduardo Alfonso Rivera }
\UCccitem{Evaluador}{}
\UCccitem{Operación}{}
\UCccitem{Prioridad}{Baja}
\UCccitem{Complejidad}{Baja}
\UCccitem{Volatilidad}{Baja}
\UCccitem{Madurez}{Alta}
\UCccitem{Estatus}{Edición}
\UCccitem{Fecha del último estatus}{17 de diciembre 2021}

% Copie y pegue este bloque tantas veces como revisiones tenga el caso de uso.
% Esta sección la debe llenar solo el Revisor
% --------------------------------------------------------

% Revisión Versión 
% Anote la versión que se revisó
\UCccsection{Revisión Versión 0.1 }

% Fecha
% Anote la fecha en que se terminó la revisión
\UCccitem{Fecha}{} 

% Evaluador
% Coloque el nombre completo de quien realizó la revisión
\UCccitem{Evaluador}{}

% Resultado
% Coloque la palabra que mas se apegue al tipo de acción que el analista debe realizar
\UCccitem{Resultado}{}

% Observaciones
% Liste los cambios que debe realizar el Analista.
\UCccitem{Observaciones}{}

% --------------------------------------------------------
	
\UCsection{Atributos}
	\UCitem{Actor(es)}
	{
		\cdtRef{Actor:RT}{Administrador de la farmacia}
	}
	\UCitem{Propósito}
	{
		Visualizar la información de una venta de la farmacia.
	}
	\UCitem{Entradas}
	{
		\UCli Número de ticket.
	}
	\UCitem{Salidas}
	{
		\UCli Estatus de la venta.
		\UCli Nombre de articulos vendidos.
		\UCli Cantidad de articulos vendidos.
		\UCli Precio por articulos.
		\UCli Número de ticket o código de venta.
		\UCli Fecha y hora de venta.
		\UCli Nombre de empleado que atendió la venta.
		\UCli Total de la venta.
	}

	\UCitem{Precondiciones}
	{
		\UCli Debe existir la venta dentro del historial de ventas.
		\UCli Se toma en cuenta la RN025 y RN026 para consultar los detalles de la venta.
		\UCli Realizar la busqueda en el historial de la venta a consultar.
	}
	\UCitem{Postcondiciones}
	{
		Ninguna
	}
	\UCitem{Reglas de negocio}
	{
		\UCli RN025: Toda venta realizada en la farmacia debera tener asociado el nombre del empleado que realiza la venta, así mismo, si un empleado realiza una modificación en una venta, el nombre del empleado se actualiza por el empleado que realizo la modificación.
		
	}
	\UCitem{Errores}
	{
		Ninguna
	}
	\UCitem{Tipo}{}
\end{UseCase}

%Trayectoria principal

\begin{UCtrayectoria}
		
	\UCpaso [\UCactor]  Realiza busqueda de una venta en historial. <<include>> CU1.3.1 Busqueda de una venta en historial.
	\UCpaso [\UCactor]  Da click en el icono de lupa en la pantalla Gestionar venta del historial. 
	\UCpaso [\UCsist] Obtiene los detalles de la venta en la BD. [Trayectoria A]
	\UCpaso [\UCsist] Muestra la pantalla2 Consultar detalles de una venta del historial, con la información obtenida de los detalles de la venta.
	
\end{UCtrayectoria}


% Trayectorias alternativas
% Trayectoria alternativa A
\begin{UCtrayectoriaA}{A}{ No puede realizar la consulta de la venta en la base de datos.}
	\UCpaso [\UCsist] Notifica con mensaje4: "No se realizo la consulta en la base de datos".
	\UCpaso [\UCsist] Se regresará a el paso 1 de la TP.
\end{UCtrayectoriaA}