
% El tipo de regla de negocio (tercer parámetro del entorno 'BusinessRule') se describe en la siguiente tabla:
%---------------------------------------------------------------------------------------------------------------,
% TIPOS			|		DEFINICION						|	EJEMPLO												|
%---------------|---------------------------------------|-------------------------------------------------------|
% Habilitador   | La sentencia habilita o restringe 	| * Se pueden recibir solicitudes del tipo A, B y C.	|
%				| hacer algo o  una funcionalidad.		| * Se permite hacer algo si se tiene el estado X.		|
%---------------|---------------------------------------|-------------------------------------------------------|
% Cronometrado	| Se permite de manera controlada 		| * Se permiten hasta dos solicitudes del tipo D		| 
%				| por un contador.						|   por persona.										|
% 				|										| * El acceso al sistema se permite si no se tiene 		|
%  				|										|   más de X número de intentos fallidos.				|
%---------------|---------------------------------------|-------------------------------------------------------|
% Ejecutivo		| Autorizado por un superior, un perfil | * Se permite registrar extemporaneamente si lo 		|
%				| particular debe autorizar.			|   autoriza X.											|
%---------------|---------------------------------------|-------------------------------------------------------|
% Derivación	| Son de cálculo e inferencia, 			| * Un alumno irregular es aquel que tiene las 			|
%				| es un cálculo o conclusión derivados 	|   siguientes cacteristicas: A, B, C. 					|
%				| de un conjunto de datos. Puede ser una| * El formato de un correo o CURP.						|
%				| fórmula que dice cómo calcular algo 	|														|
%				| o el formato de un dato.				|														|
%---------------|---------------------------------------|-------------------------------------------------------|
% Restricción	| Restringe una funcionalidad o relación| * Traslape de fechas o periodos empalmados.			|
% 				| entre dos o mas objetos.				|														|
%  				|										|														|
%---------------------------------------------------------------------------------------------------------------'

% No editar las reglas cuyo estatus es APROBADO.
\section{Reglas de negocio}
\subsection{Reglas derivadas del sistema}
%------------------------------------------------------------------------------------------------------------------
\begin{comment}
\begin{BusinessRule}{RN-1}{Definición de producto}
	{Definición}
	{Controla la operación}
	\BRitem{Versión}{1.0}
	\BRitem{Autor}{Oscar Apolonio Tierrablanca}
	\BRitem{Estatus}{Edición}
	\BRitem{Descripción}{Un prodcuto es asquel que tiene un nombre, un código de barras y una descripción asosiados y se encuentra en la base de datos.}
		
	
\end{BusinessRule}

\begin{BusinessRule}{RN-2}{Formato de codigo de barras.}
	{Restricción}
	{Controla la operación}
	\BRitem{Versión}{1.0}
	\BRitem{Autor}{Oscar Apolonio Tierrablanca}
	\BRitem{Estatus}{Edición}
	\BRitem{Descripción}{El codigo de barras es una cadena de numeros de 13 caracteres}
		
	
\end{BusinessRule}

\begin{BusinessRule}{RN-3}{Venta con receta médica.}
	{Restricción}
	{Controla la operación}
	\BRitem{Versión}{1.0}
	\BRitem{Autor}{Jair Lechuga}
	\BRitem{Estatus}{Edición}
	\BRitem{Descripción}{
		El medicamento antibiótico, medicamento controlado y algunos otros que necesiten previa autorización de un médico no podrán venderse sin receta médica valida.
	}
		
\end{BusinessRule}

\begin{BusinessRule}{RN-4}{Eliminar proveedor con pedidos pendientes}
	{Restricción}
	{Controla la operación}
	\BRitem{Versión}{1.0}
	\BRitem{Autor}{Gonzaga Aparicio Josue}
	\BRitem{Estatus}{Revisiòn}
	\BRitem{Descripción}{
		Un proveedor con pedidos pendientes no puede ser eliminado. (Eliminaciòn lògica).
	}
		
\end{BusinessRule}

\begin{BusinessRule}{RN-6}{Cantidad de productos en carrito valida.}
	{Restricción}
	{Controla la operación}
	\BRitem{Versión}{1.0}
	\BRitem{Autor}{Reyes Vaca Mauricio Alberto}
	\BRitem{Estatus}{Edición}
	\BRitem{Descripción}{
		La cantidad debe de ser mayor a 0 y menor o igual a cantidad en existencia.
	}
		
\end{BusinessRule}

\begin{BusinessRule}{RN-07}{Asociacion de productos a proveedor}
	{Restricción}
	{Controla la operación}
	\BRitem{Versión}{1.0}
	\BRitem{Autor}{Apolonio Tierrablanca Oscar}
	\BRitem{Estatus}{Edición}
	\BRitem{Descripción}{Todos los productos deben estar asociados a un proveedor.}

\end{BusinessRule}

\end{comment}