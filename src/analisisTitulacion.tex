 \documentclass[10pt]{book}
\usepackage{cdt/cdtBusiness}
\usepackage{eld}
\usepackage{subfigure}
\usepackage{cite}
\usepackage{verbatim} 
\usepackage{mdframed} 
\usepackage{authblk}
%%%%%%%%%%%%%%%%%%%%%%%%%%%%%%%%%%%%%%%%%%%%%%%%%%%%%%%%%%%%%%%%
% Datos del proyecto

\organizacion[ESCOM]{Escuela Superior de Cómputo}
\author{Reyes Mauricio$^{1}$, Sevillano Victor$^{2}$, Montalvo Ana$^{3}$}
\sistema[Transporte escolar]{Prueba}
\proyecto[App]{Trabajo Terminal I}
\documento{DocLaTeX}{App de transporte escolar alternativo para la comunidad estudiantil ESCOM}{\RELEASE{1.0}}% {\DRAFT{\today}} 

\entregable{}{\Large{Dr.Cifuentes Álvarez Alejandro Sigfrido}}


%\docRefs{
%	
%	%     \docItem{Catálogo de Escuelas}{}{Catálogo de Escuelas de la DGAIR (Dirección General de Acreditación, Incorporación y Revalidación) de la SEP (Secretaría de Educación Pública)}
%	\docItem{Plan de Proyecto}{1.0}{ \cdtLabel{planProyecto}{Plan de proyecto del Programa de Acreditación de Escuelas Ambientalmente Responsables}}
%	\docItem{PAEAR}{1.0}{ \cdtLabel{PAEAR}{Programa de Acreditación de Escuelas Ambientalmente Responsables}}
%	\docItem{MPOCCT}{1.0}{ \cdtLabel{cct}{Manual de Procedmientos para la Operación del Catálogo de Centros de Trabajo}}
%	%     \docItem{M-1TR}{1.0}{ \cdtLabel{M-3TR}{Minuta de la Primera Reunión de Toma de Requerimientos}}
%	%     \docItem{M-2TR}{1.0}{ \cdtLabel{M-2TR}{Minuta de la Segunda Reunión de Toma de Requerimientos}}
%	\docItem{M-3TR}{1.0}{ \cdtLabel{M-3TR}{Minuta de la Tercera Reunión de Toma de Requerimientos}}
%}

%%%%%%%%%%%%%%%%%%%%%%%%%%%%%%%%%%%%%%%%%%%%%%%%%%%%%%%%%%%%%%%%
% Elementos contenidos en el documento

% TODO: Al finalizar el análisis resuma aquí todos los elementos del componente: RN, CU, IU, MSG.
\elemRefs{
	
	%Glosario de términos
	%\elemItem{Glosario de términos}{1.0}{Descripción de los terminos técnicos y de negocio utilizados}
	%---------------------------------------------------------------------------------------------------------------------------------------
	
	%Modelo de información
	%\elemItem{Modelo Registro de escuelas}{1.0}{Descripción del modelo de información del Registro de escuelas}
	%\elemItem{Modelo Periodos}{1.0}{Descripción del modelo de información de periodos de consumo de energía y agua}
	
	%---------------------------------------------------------------------------------------------------------------------------------------
	
	%Actores
	%\elemItem{Coordinador del programa}{1.0}{Descripción del actor}
	%---------------------------------------------------------------------------------------------------------------------------------------
	
	%Reglas de negocio
	%\elemItem{RN-S1}{1.0}{Información correcta}
	%---------------------------------------------------------------------------------------------------------------------------------------
	
	%Mensajes
	%\elemItem{MSG1}{1.0}{Operación realizada exitosamente}
	
	%---------------------------------------------------------------------------------------------------------------------------------------
	
	%%%%%%%%%%%%%%%%%%%
	%% CASOS DE USO %%%
	%%%%%%%%%%%%%%%%%%%
	
	%%%% REGISTRO DE ESCUELAS %%%%
	%\elemItem{CUR 1}{1.0}{Iniciar sesión}
	
	%%%%%%%%%%%%%%%%%%%
	%%%% PANTALLAS %%%%
	%%%%%%%%%%%%%%%%%%%
	
	%REGISTRO DE ESCUELAS
	%\elemItem{MN1}{1.0}{Menú del Director del programa}
	%\elemItem{MN2}{1.0}{Menú del Coordinador del programa}
	%\elemItem{IUR 1}{1.0}{Iniciar sesión}
}

%%%%%%%%%%%%%%%%%%%%%%%%%%%%%%%%%%%%%%%%%%%%%%%%%%%%%%%%%%%%%%%%
\begin{document}
	
	%=========================================================
	% Portada
	\ThisLRCornerWallPaper{1}{cdt/theme/agua.jpeg}
	\thispagestyle{empty}
	
	\maketitle
	
	%=========================================================
	% Hoja de revisión
	\makeDocInfo
	\vspace{0.5cm}
	\makeElemRefs
	\makeDocRefs
	\makeObservaciones[3cm] 
	\vspace{0.5cm}
	\makeFirmas
	
	%=========================================================
	% Indices del documento
	\frontmatter
	\LRCornerWallPaper{1}{cdt/theme/pleca.jpg}
	\tableofcontents
	%\listoffigures
	%\listoftables
	\mainmatter
	
	% Para esconder la información del documentador se descomenta el \hideControlVersion
	%\hideControlVersion
	
	%=========================================================
	\chapter{Introducción}\label{chp:introduccion}
	\cfinput{introduccion/introduccion}
	

	%=========================================================
	%\chapter{Glosario de términos}\label{chp:glosarioTerminos}
	
	%=====================MODELO DE NEGOCIO==============================================================
	\chapter{Modelo de negocio}\label{chp:modeloNegocios} 
	\cfinput{ModeloNegocios/modelo}
	\cfinput{ModeloNegocios/reglas}
	
	%=====================MODELO DE NEGOCIO==============================================================
	\chapter{Pantallas}\label{chp:pantallas}
	\cfinput{pantallas/pantallas}
	
	%=====================MODELO DE NEGOCIO==============================================================
	\chapter{Proceso}\label{chp:proceso}
	\cfinput{proceso/proceso}

	%=====================MODELO DE COMPORTAMIENTO=======================================================
	\chapter{Modelo de comportamiento}\label{chp:modeloComportamiento}
	%\cfinput{casouso/cutemplate/uc}
	%modulo1
	\cfinput{casouso/cu1/uc}
	\cfinput{casouso/cu2/uc}
	\cfinput{casouso/cu5/uc}
	\cfinput{casouso/cu6/uc}
	\cfinput{casouso/cu8/uc}
	\cfinput{casouso/cu10/uc}
	\cfinput{casouso/cu23/uc}
	\cfinput{casouso/cu24/uc}
	\cfinput{casouso/cu25/uc}
	\cfinput{casouso/cu26/uc}
	%\section{Diseño de mensajes}
	%\cfinput{ModeloInteraccion/mensajes}

\end{document}